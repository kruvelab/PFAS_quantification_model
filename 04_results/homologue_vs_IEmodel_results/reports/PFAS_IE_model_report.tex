% Options for packages loaded elsewhere
\PassOptionsToPackage{unicode}{hyperref}
\PassOptionsToPackage{hyphens}{url}
%
\documentclass[
]{article}
\usepackage{amsmath,amssymb}
\usepackage{lmodern}
\usepackage{ifxetex,ifluatex}
\ifnum 0\ifxetex 1\fi\ifluatex 1\fi=0 % if pdftex
  \usepackage[T1]{fontenc}
  \usepackage[utf8]{inputenc}
  \usepackage{textcomp} % provide euro and other symbols
\else % if luatex or xetex
  \usepackage{unicode-math}
  \defaultfontfeatures{Scale=MatchLowercase}
  \defaultfontfeatures[\rmfamily]{Ligatures=TeX,Scale=1}
\fi
% Use upquote if available, for straight quotes in verbatim environments
\IfFileExists{upquote.sty}{\usepackage{upquote}}{}
\IfFileExists{microtype.sty}{% use microtype if available
  \usepackage[]{microtype}
  \UseMicrotypeSet[protrusion]{basicmath} % disable protrusion for tt fonts
}{}
\makeatletter
\@ifundefined{KOMAClassName}{% if non-KOMA class
  \IfFileExists{parskip.sty}{%
    \usepackage{parskip}
  }{% else
    \setlength{\parindent}{0pt}
    \setlength{\parskip}{6pt plus 2pt minus 1pt}}
}{% if KOMA class
  \KOMAoptions{parskip=half}}
\makeatother
\usepackage{xcolor}
\IfFileExists{xurl.sty}{\usepackage{xurl}}{} % add URL line breaks if available
\IfFileExists{bookmark.sty}{\usepackage{bookmark}}{\usepackage{hyperref}}
\hypersetup{
  hidelinks,
  pdfcreator={LaTeX via pandoc}}
\urlstyle{same} % disable monospaced font for URLs
\usepackage[margin=1in]{geometry}
\usepackage{graphicx}
\makeatletter
\def\maxwidth{\ifdim\Gin@nat@width>\linewidth\linewidth\else\Gin@nat@width\fi}
\def\maxheight{\ifdim\Gin@nat@height>\textheight\textheight\else\Gin@nat@height\fi}
\makeatother
% Scale images if necessary, so that they will not overflow the page
% margins by default, and it is still possible to overwrite the defaults
% using explicit options in \includegraphics[width, height, ...]{}
\setkeys{Gin}{width=\maxwidth,height=\maxheight,keepaspectratio}
% Set default figure placement to htbp
\makeatletter
\def\fps@figure{htbp}
\makeatother
\setlength{\emergencystretch}{3em} % prevent overfull lines
\providecommand{\tightlist}{%
  \setlength{\itemsep}{0pt}\setlength{\parskip}{0pt}}
\setcounter{secnumdepth}{-\maxdimen} % remove section numbering
\ifluatex
  \usepackage{selnolig}  % disable illegal ligatures
\fi

\author{}
\date{\vspace{-2.5em}}

\begin{document}

\hypertarget{semi-quantification-model-development}{%
\subsection{Semi-quantification: model
development}\label{semi-quantification-model-development}}

All existing data was used to train IE prediction model for ESI-. For
this, training and test sets of Thomas' data were joined. This model can
be used for any semi-quantification of PFAS.

The code has been cleaned and organized in functions for smooth
retraining if additional data is available for updating the model. The
code and the model can be found from GitHub:
\url{https://github.com/kruvelab/PFOA_semi_quant}

\hypertarget{semi-quantification-homologous-series-vs-ml-model-predictions}{%
\subsection{Semi-quantification: Homologous series vs ML model
predictions}\label{semi-quantification-homologous-series-vs-ml-model-predictions}}

In addition to updating the IE prediction model, homologue series was
tested for semi-quantification. For this, the following steps were
taken:

\begin{enumerate}
\def\labelenumi{\arabic{enumi})}
\tightlist
\item
  Finding homologue series compounds
\item
  Computing the calibration graph for all of the chemicals
\item
  Assigning the calibration graph of the closest homologue to each
  analyte
\item
  Semi-quantification with the calibration graph of the homologue This
  was repeated for all chemicals in the homologue series.
\end{enumerate}

In addition the IE prediction model was used under comparable
conditions. For this, following steps were taken: 1) Training IE
predictive model based on all of the analytical standard, excluding the
analyte 2) Predicting ionization efficiency for the analyte with the
trained IE model 3) Predicting response factors from the predicted
ionization efficiency by using RF vs IE plot of all analytical standards
4) Semi-quantification based on the predicted response factors

Finally, the results from the two approaches were compared visually and
numerically.

\hypertarget{finding-homologue-series-compounds}{%
\subsubsection{Finding homologue series
compounds}\label{finding-homologue-series-compounds}}

From Thomas' data, a dataset was generated, which contained all
compounds that had at least one homologue within the dataset.

\textbf{Assumption:} Two compounds are considered homologues when their
difference in molecular formula is CF\textsubscript{2}.

This summary is made on the example of CF\textsubscript{2} homologues
only, see below. Data for CF\textsubscript{2}CF\textsubscript{2}
homologues can be calculated if needed.

\begin{verbatim}
## # A tibble: 18 x 3
##    Compound Homologue pattern_CF2
##    <chr>    <chr>     <chr>      
##  1 PFDA     PFNA      smaller    
##  2 PFDA     PFUnDA    bigger     
##  3 PFDoDA   PFTriDA   bigger     
##  4 PFDoDA   PFUnDA    smaller    
##  5 PFHpA    PFHxA     smaller    
##  6 PFHpA    PFOA      bigger     
##  7 PFHxA    PFHpA     bigger     
##  8 PFHxA    PFPeA     smaller    
##  9 PFNA     PFDA      bigger     
## 10 PFNA     PFOA      smaller    
## 11 PFOA     PFHpA     smaller    
## 12 PFOA     PFNA      bigger     
## 13 PFPeA    PFHxA     bigger     
## 14 PFTeDA   PFTriDA   smaller    
## 15 PFTriDA  PFDoDA    smaller    
## 16 PFTriDA  PFTeDA    bigger     
## 17 PFUnDA   PFDA      smaller    
## 18 PFUnDA   PFDoDA    bigger
\end{verbatim}

\hypertarget{semi-quantification-with-homologue}{%
\subsubsection{Semi-quantification with
homologue}\label{semi-quantification-with-homologue}}

For each compound, calibration curve of a homologue was used for
semi-quantification. If two homologues existed (bigger and smaller),
quantification was done with both and both results are shown on the
graph as well as taken into account in performance calculations.

The IE approach assumes that the intercept of the calibration graph is
statistically insignificant or much smaller than the peak areas.
Therefore, two different approaches were also used for the
quantification with the homologue series.

\textbf{First approach:} Only slope (RF) was used to calculate
concentrations (regression line was not forced to go though zero).

(\(conc = area/slope_{homologue}\))

\textbf{Second approach:} Both slope and intercept were used to
calculate concentrations.

(\(conc = (area-intercept_{homologue})/slope_{homologue}\))

\hypertarget{predicting-ionization-efficiency-and-response-factors-for-homologue-series-compounds-quantification}{%
\subsubsection{Predicting ionization efficiency and response factors for
homologue series compounds +
quantification}\label{predicting-ionization-efficiency-and-response-factors-for-homologue-series-compounds-quantification}}

For each homologue series compound, the compound was removed from the
training data and prediction model was trained (10 prediction models
were trained in total). Then, the model was used to predict IE of the
compound. The predicted ionization efficiency values were correlated
with experimental response factors (that is slope) for all analytical
standards, this correlation was used to predict a response factor for
the analyte from the IE.

(\(conc = area/slope_{predicted}\))

\hypertarget{visual-comparison}{%
\subsubsection{Visual comparison}\label{visual-comparison}}

Comparing semi-quantification results from predicted slopes and
homologue series compounds slopes with theoretical concentration. Ideal
regression and ten-times error lines were added.

\includegraphics{PFAS_IE_model_report_files/figure-latex/unnamed-chunk-2-1.pdf}

\hypertarget{results-and-interpretation}{%
\subsection{Results and
interpretation}\label{results-and-interpretation}}

The error factor for IE and homologue series, first approach, based
semi-quantification are very similar ranging from 2.3x to 3.0x.
Homologue series based approach shows slightly higher error factor;
therefore, it can be concluded that a single IE based
semi-quantification can be used for all PFAS and homologue series based
quantification is not necessary. See results below. Importantly, for
both methods we see that the quantification is less accurate for lower
concentration level. This probably arises from slightly reduced
linearity at low concentration and/or importance of intercept at these
low levels.

\begin{verbatim}
## # A tibble: 2 x 3
##   pattern error_IE error_homolog
##   <chr>      <dbl>         <dbl>
## 1 bigger      2.44          2.67
## 2 smaller     2.27          2.96
\end{verbatim}

\end{document}
